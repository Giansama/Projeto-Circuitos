\documentclass[10pt]{beamer}
\usetheme[
%%% options passed to the outer theme
%    hidetitle,           % hide the (short) title in the sidebar
%    hideauthor,          % hide the (short) author in the sidebar
%    hideinstitute,       % hide the (short) institute in the bottom of the sidebar
%    shownavsym,          % show the navigation symbols
%    width=2cm,           % width of the sidebar (default is 2 cm)
%    hideothersubsections,% hide all subsections but the subsections in the current section
%    hideallsubsections,  % hide all subsections
%    left                % right of left position of sidebar (default is right)
  ]{Aalborg}
  
% If you want to change the colors of the various elements in the theme, edit and uncomment the following lines
% Change the bar and sidebar colors:
%\setbeamercolor{Aalborg}{fg=red!20,bg=red}
%\setbeamercolor{sidebar}{bg=red!20}
% Change the color of the structural elements:
%\setbeamercolor{structure}{fg=red}
% Change the frame title text color:
%\setbeamercolor{frametitle}{fg=blue}
% Change the normal text color background:
%\setbeamercolor{normal text}{bg=gray!10}
% ... and you can of course change a lot more - see the beamer user manual.

\usepackage[utf8]{inputenc}
\usepackage[portuguese]{babel}
\usepackage[T1]{fontenc}
% Or whatever. Note that the encoding and the font should match. If T1
% does not look nice, try deleting the line with the fontenc.
\usepackage{helvet}


% colored hyperlinks
\newcommand{\chref}[2]{%
  \href{#1}{{\usebeamercolor[bg]{Aalborg}#2}}%
}

\title[Jogo da Velocidade]% optional, use only with long paper titles
{Projeto de Circuitos Digitais}

\subtitle{Jogo da Velocidade}  % could also be a conference name

\date{\today}

\author[Unidade III] % optional, use only with lots of authors
{
  Gian Lucas Cavalcante de Lima \\
  \and 
  Leandro Augusto Barbosa \\
  \and 
  Rafael de Medeiros Mariz Capuano \\
  \and 
  Thiago de Oliveira Nunes Galeno \\
  \and 
  Tiago Aleixo de Araújo \\
  \and 
  Wysterlanya Kyury Pereira Barros
}
% - Give the names in the same order as they appear in the paper.
% - Use the \inst{?} command only if the authors have different
%   affiliation. See the beamer manual for an example

\institute[
%  {\includegraphics[scale=0.2]{aau_segl}}\\ %insert a company, department or university logo
  Dept.\ de Engenharia de Computação e Automação\\
  UFRN
] % optional - is placed in the bottom of the sidebar on every slide
{% is placed on the bottom of the title page
  Departamento de Engenharia de Computação e Automação\\
  Universidade Federal do Rio Grande do Norte
  
  %there must be an empty line above this line - otherwise some unwanted space is added between the university and the country (I do not know why;( )
}

% specify the logo in the top right/left of the slide
\pgfdeclareimage[height=0.7cm]{mainlogo}{AAUgraphics/dca} % placed in the upper left/right corner
\logo{\pgfuseimage{mainlogo}}

% specify a logo on the titlepage (you can specify additional logos an include them in 
% institute command below
\pgfdeclareimage[height=1.6cm]{titlepagelogo}{AAUgraphics/brasaoufrn} % placed on the title page
%\pgfdeclareimage[height=1.5cm]{titlepagelogo2}{AAUgraphics/aau_logo_new} % placed on the title page
\titlegraphic{% is placed on the bottom of the title page
  \pgfuseimage{titlepagelogo}
%  \hspace{1cm}\pgfuseimage{titlepagelogo2}
}

\begin{document}
% the titlepage
{\aauwavesbg
\begin{frame}[plain,noframenumbering] % the plain option removes the sidebar and header from the title page
  \titlepage
\end{frame}}
%%%%%%%%%%%%%%%%

% TOC
\begin{frame}{Sumário}{}
\tableofcontents
\end{frame}
%%%%%%%%%%%%%%%%

\section{Introdução}
\begin{frame}{Introdução}{}
\begin{block}{O que é o projeto?}
  \begin{itemize}
    \item<1-> O projeto é um trabalho em grupo aonde será desenvolvido um jogo para dois jogadores escrito em VHDL com a finalidade de por em prática todo o conteúdo aprendido durante o semestre, praticar trabalho em grupo e avaliar os resultados do trabalho.
  \end{itemize}
\end{block}
\end{frame}
%%%%%%%%%%%%%%%%

\subsection{Objetivo}
% the license
\begin{frame}{Introdução}{Objetivo}
  \begin{itemize}
    \item<1-> O objetivo do jogo é ser um jogo de memória e reflexos aonde o primeiro jogador entra com uma sequência de 5 digitos em 3 botões e o segundo jogador deve repetir a sequência corretamente no menor tempo possível.
  \end{itemize}
\end{frame}
%%%%%%%%%%%%%%%%

\subsection{O jogo}
\begin{frame}{Introdução}{O jogo}
  \begin{itemize}
    \item<1-> O objetivo do jogo é ser um jogo de memória e reflexos aonde o primeiro jogador entra com uma sequência de 5 digitos em 3 botões e o segundo jogador deve repetir a sequência corretamente no menor tempo possível.
  \end{itemize}
\end{frame}
%%%%%%%%%%%%%%%%

\section{Desenvolvimento}
% general installation instructions
\begin{frame}{Desenvolvimento}
  "Explicar como foi feito o trabalho e demais informações do desenvolvimento."
\end{frame}

\begin{frame}{Desenvolvimento}{Máquina de estados de alto nível}
  "imagem da maquina de estados aqui."
\end{frame}

\section{Funcionamento}
% general installation instructions
\begin{frame}{Funcionamento}
  "Explicar como funciona o programa, e como cada parte dele funciona e com que objetivo ele faz tal tarefa."
\end{frame}


\section{Considerações finais}
\subsection{Informações dos componentes}
% contact information
\begin{frame}{Finalizações}{Informações dos componentes}

\begin{tabular}{lr}
  Gian Lucas Cavalcante de Lima & 2012919250\\
  Leandro Augusto Barbosa & 2010038622 \\
  Rafael de Medeiros Mariz Capuano & 2012921679 \\
  Thiago de Oliveira Nunes Galeno & 2011011884  \\
  Tiago Aleixo de Araújo & 2013023516  \\
  Wysterlanya Kyury Pereira Barros & 2014028947  \\
\end{tabular}

\end{frame}
%%%%%%%%%%%%%%%%

{\aauwavesbg%
\begin{frame}[plain,noframenumbering]%
  \finalpage{Agora vamos por o jogo em ação!}
\end{frame}}
%%%%%%%%%%%%%%%%

\end{document}
